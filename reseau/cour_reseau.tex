\documentclass[a4paper,12pt,openany]{book}
\usepackage [utf8]{inputenc}
\usepackage [french]{babel}
\usepackage [T1]{fontenc}
\usepackage{listings}
\usepackage{graphicx}
\usepackage{verbatim}

%%configuration de listings
\lstset{
language=c,
basicstyle=\ttfamily\small,
identifierstyle=\color{red},
keywordstyle=\color{blue},
stringstyle=\color{black!60},
commentstyle=\it\color{green!95!yellow!1},
columns=flexible,
tabsize=1,
extendedchars=true,
showspaces=false,
showstringspaces=false,
numbers=left,
numberstyle=\tiny,
breaklines=true,
breakautoindent=true,
captionpos=b
}

%coloration syntaxique
\usepackage{xcolor}
\definecolor{Zgris}{rgb}{0.87,0.85,0.85}

%\newsavebox{\BBbox}
%\newenvironment{DDbox}[1]{
%\begin{lrbox}{\BBbox}\begin{minipage}{\linewidth}}
%{\end{minipage}\end{lrbox}\noindent\colorbox{Zgris}{\usebox{\BBbox}}
%[.5cm]}

%Pour l espace entre la section et la chapitre (qui est trop grand).
\usepackage{titlesec}

\titleformat{\chapter}[block]
  {\normalfont\Huge\bfseries}% font of number
  {\chaptertitlename\ \thechapter~:}% format of number
  {20pt}% space between number and title
  {\Huge}% font of title

\titlespacing*{\chapter}
  {0pt}%  indent
  {0pt}% space before
  {20pt}% space after
\titlespacing*{\section}
  {0pt}%  indent
  {3.5ex plus 1ex minus .2ex}% space before
  {2.3ex plus .2ex}% space after

\author{Mendy Fatnassi}
\title{Cours de Reseau}

%%%%%%%%%%%%%%%%%%%%%%%%%%%%%%%%%%%%%%	Page	%%%%%%%%%%%%%%%%%%%%%%%%%%%%%%%%%%%%%%%%
\begin{document}
\maketitle
\tableofcontents

\chapter{Generalité}
\section{Vocabulaire}
On appelle reseau informatique tout ce qui permet d'echanger des donnees entre des stations (ordinateur,telephone...).\\
Voici quelque protocole : \\
-FTP (File Transfert Proctocol) et SMTP (Simple Mail Transfert Protocole) -> ces 2 protocole ne sont pas sécuriser , les messages apparaissent en claire .\\
-DNS(Domain Name System) et TCP/IP qui est le protocol le plus utilise (langage universel) .\\
\\
\underline{Serveur} : Heberge les services (stockage,impression...).\\
\underline{Client} : S'adressent aux serveur pour beneficier des services.\\
\underline{Peer-To-Peer} : Chaque station est client/serveur.\\
\\
Un \textbf{medias reseau} designe un support physique pour l'echange des donnees (cable,switch,routeur).\\
Un reseau peut etre fermé c'est a dire en locale on perlera de \textbf{LAN (Local Area Network)} on pourra le controler directement et le configurer . A contrario un reseau ouvert a tout le monde est le \textbf{WAN (Worl Area Networck)} concernet tous les autres reaseau et est non-configurable.\\

\underline{Internet} : Interconnection de tous les reseau .\\
\underline{Intranet} : Reseau privée dedié a l'usage exclusif d'une entreprise ,besoins d'une authentification pour acceder au service du reseau.\\
\underline{Extranet} : Reseau public , service accessible a l'exterieur du reseau de l'entreprise .\\
\underline{BYOD(Bring Your Own Device)} : Permet a l'utilisateur d'utilisé ses propres equipements dans l'entreprise.\\
\\

\underline{Routeur}
Principale mode de transport la commutation .Chaque paquet contient l'adresse de l'emetteur et du destinataire . Chaque noeudpossede une table de routage.\\
Un reseau informatique est \textbf{maillé} grace a des \textbf{noeuds} , l'information circule sous forme numerique , le flux est segmenter c'est ce qu'on appelle "paquet" .\\
La communication numerique a besoins d'encodage et decodage du message , on utilise aussi l'encapsulation pour ajouter de l'information au message (emmetteur,date...) .\\
\\
Une communication peut etre :\\
-Unicast : 1 emmetteur communique avec 1 recepteur .\\
-Broadcast :  1 emmeteur communique avec plusieur recepteur.\\



\chapter{Modele OSI}
\section{Generalite du modele OSI}
OSI(Open System Internconnection) : Permet de definir comment les appereils vont communiquer entre eux.\\
ISO (International Organisation for Standardization) : A pas confondre avec OSI .\\
\\
Le modele OSI possede 7 couches :\\
-1(Physique): Transmission des signaux (numerique ou analogique).\verb+\bits+\\
-2(Liaison): Adessage physique(MAC).\verb+\trame+\\
-3(Reseau): Determine le parcours des donnees.\verb+\Paquet+\\
-4(Transport): (TCP et UDP) controle-flux , peer-to-peer.\verb+\Datagramme+\\
-5(Session)\\
-6(Presentation)\\
-7(Application)\\
\\
Voici un schema :\\
\includgraphics[width=1\textwidth]{osi_modele.png}
Le transport des donnees se fait grace aux couches 1(physique),2(liaison),3(reseau) et 4(transport) . Le traitement des données se fait avec les autres couches 5(session),6(presentation) et 7(application).\\
Le \textbf{multiplexage}(couche 4.transport) : heberger plusieurs applicatifs sur un meme hote.\\
Le datagramme et le paquet (couches 4 et 3) sont les entites qui reste identique tous le long du trajet.\\

\section{1.Physique}
La couche physique concerne le cablage du reseau , a l'aide de cable RJ45.\\
-Paire torsadée non-blindé UTP (100 Mbits/s).\\
-Paire torsadée blindé STP,FTP (+rapide,+longue distance que UTP).\\
\\
On utilise des cable droit pour les appareils de niveau 1 et 2 du modele OSI sinon on utilise un cable paire-croisé (fait automatiquement par le switch). 
\\
Materiel :\\
-\underline{HUB} : Niveau/Couche 1 , Permet de transmettre l'information a tous les utilisateurs . Permet de regenerer un signal entre deux segments(paquet) du reseau .\\
-\underline{Switch} : Niveau 2 , Permet de diviser un LAN en plusieurs segments .\\
-\underline{Routeur} : Niveau 3 , Permet de definir l'acheminement des paquets .\\


\section{2.Laison}
Definit la topologie du reseau , soit en etoile,maillé ou en bus.Fourni aux couches superieur le moyen de controller l'acces aux medias reseau et detecter les erreurs de transmission sur le media .\\
possede 2 sous couche MAC et LLC , necessite l'etude de la trame ARP .\\
Dispose de hub et switch .\\

\section{3.Reseau}
Permet de definir l'adresse IP de la machine , en utilisant un protocol IPv4 ou IPv6 .\\
Ajout de l'information pour former les paquet (encapsulation).\\
\\
\underline{Routage} : Par quels equipements de niveau 3 les paquets vont passe pour atteindre le reseau de destination (table de routage NAT) .\\
\\
L'attribution des adresse peux se faire de 2 facons :\\
-Statique : Configuration explicite/manuelle (IP,Masque...).\\
-Dynamique : DHCP pour les reseau de grande taille et Les reseaux ou les utilisateur changent souvent.

\section{4.Transport}
\section{5.Session}
\section{6.Presentation}
\section{7.Appication}




\end{document}
