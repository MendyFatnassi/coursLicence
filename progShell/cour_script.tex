\documentclass[a4paper,12pt]{book}

\usepackage [utf8]{inputenc}
\usepackage [french]{babel}
\usepackage [T1]{fontenc}


\author{Mendy Fatnassi}
\title{Linux Programmation Shell}

\begin{document}
\maketitle
\chapter{Script Bash}
\\
Les différents environnements console sont appelés des shells, en voiciquelques uns :\\
\\
sh : Bourne Shell. L'ancêtre de tous les shells.\\
\\
bash : Bourne Again Shell. Une amélioration du Bourne Shell, disponible par défaut sous Linux et Mac OS X.\\
\\
les fichier script porte l'extension nomscript.sh .\\
\\
\section{Structure code}
En-tete du script(obligation) : #!/bin/bash\\
\\
Deboger son script : bash -x nomscript.sh\\
\\
Afficher une variable : \\
mavar = "bonjour le monde!"
echo \$mavar
\\
Pour demander une saisie on utilise \textbf{read} , option -p pour mettre un message : \\
read "entrez nom" nom \\
echo "bonjour \$nom" \\
\\
Parametre : \$1 \$2 \$3 etc...\\
\\
\underline{Structure de controle} :\\
Pour comparer des entiers -eq(equal),-ne(different de),lt(inferieur),le(inf ou egal),gt(greater than),ge(greater equal).\\
Pour comparer 2 string if [ "ma string" = "\$1" ] .\\
\\
if [ condition ]
then
...
fi


\section{Crontab : execution differer}
\\
La crontab est un outil qui nous permet de programmer l'execution réguliere d'un programme.Il existe une crontab par utilisateur .
\\
-e : modifier la crontab ;\\
-l : afficher la crontab actuelle ;\\
-r : supprimer votre crontab. Attention, la suppression est immédiate et sans confirmation !\\
\\
Utilisation crontab: # m h  dom mon dow   command
\\
Comme cette ligne est précédée d'un #, il s'agit d'un commentaire (qui sera donc ignoré).\\
Cette ligne vous donne quelques indications sur la syntaxe du fichier :\\
\\
-m : minutes (0 - 59) ;\\
-h : heures (0 - 23) ;\\
-dom (day of month) : jour du mois (1 - 31) ;\\
-mon (month) : mois (1 - 12) ;\\
-dow (day of week) : jour de la semaine (0 - 6, 0 étant le dimanche);\\
-command : c'est la commande à exécuter.\\
\\
Quelque notation :\\
\\
-5 (un nombre) : exécuté lorsque le champ prend la valeur 5 ;\\
-* : exécuté tout le temps (toutes les valeurs sont bonnes) ;\\
-3,5,10 : exécuté lorsque le champ prend la valeur 3, 5 ou 10. Ne pas mettre d'espace après la virgule ;\\
-3-7 : exécuté pour les valeurs 3 à 7 ;\\
-*/3 : exécuté tous les multiples de 3 (par exemple à 0 h, 3 h, 6 h, 9 h…).\\
\\
Exemple : \\
-30 5 1-15 * * commande -> a 5 h 30 du matin du 1er au 15 de chaque mois.\\
-0 0 * * 1,3,4 commande -> a minuit le lundi, le mercredi et le jeudi.\\
-0 */2 * * * commande -> Toutes les 2 heures (00 h 00, 02 h 00, 04 h 00…).\\
-*/10 * * * 1-5 commande -> Toutes les 10 minutes du lundi au vendredi.\\
\\
\end{document}
