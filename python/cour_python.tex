\documentclass[a4paper,12pt,openany]{book}
\usepackage [utf8]{inputenc}
\usepackage [french]{babel}
\usepackage [T1]{fontenc}
\usepackage{graphicx}
\usepackage{titlesec}
\usepackage{listings}


%%configuration de listings
\lstset{
language=python,
basicstyle=\ttfamily\small, 
identifierstyle=\color{red}, 
keywordstyle=\color{blue}, 
stringstyle=\color{black!60}, 
commentstyle=\it\color{green!95!yellow!1}, 
columns=flexible, 
tabsize=1, 
extendedchars=true, 
showspaces=false, 
showstringspaces=false, 
numbers=left, 
numberstyle=\tiny, 
breaklines=true, 
breakautoindent=true, 
captionpos=b
}

%coloration syntaxique
\usepackage{xcolor}
\definecolor{Zgris}{rgb}{0.87,0.85,0.85}


\titleformat{\chapter}[block]
  {\normalfont\Huge\bfseries}% font of number
  {\chaptertitlename\ \thechapter~:}% format of number
  {20pt}% space between number and title
  {\Huge}% font of title

\titlespacing*{\chapter}
  {0pt}%  indent
  {0pt}% space before
  {20pt}% space after
\titlespacing*{\section}
  {0pt}%  indent
  {3.5ex plus 1ex minus .2ex}% space before
  {2.3ex plus .2ex}% space after

\author{Mendy Fatnassi}
\title{Cours de Programmation en Python}

%%%%%%%%%%%%%%%%%%%%%%%%%%%%%%%%%%%%%%	Page	%%%%%%%%%%%%%%%%%%%%%%%%%%%%%%%%%%%%%%%%
\begin{document}
\maketitle
\tableofcontents


\chapter{Introduction}
\section{Pr\'esentation}
\\
Python est un langage de programmation interpr\'et\'e, c est-à-dire que les instructions que vous lui envoyez sont \verb+« transcrites »+ en langage machine au fur et à mesure de leur lecture.\\
 D'autres langages (comme le C / C++) sont appelés « langages compil\'es » car, avant de pouvoir les ex\'ecuter, un logiciel spécialisé se charge de transformer le code du programme en langage machine.\\
On appelle cette \'etape la « compilation ».\\

\section{Installation bibliotheque python}
Depuis python 3.4 l'utilitaire \'pip\' est installé avec la package d'installation , sinon telecharger le : \$sudo apt\-get install pip \\
Une fois cela fait utilisé  \'pip\' pour telecharger les bibliotheque puthon : \$sudo pip install nom\_biblio \\

\section{Python Generalité}

Comme pour un script sh , python a un shebang \verb+"#!/usr/bin/python"+ .\\
Pour lancer un l'interpreteur python depuis le terminal tapez : pythonX.X (python3.5).\\
Pour lancer directement un script depuis le terminal : \verb + $python nom_script.py OU ./nom_script.py +\\

%%%%%%%%%%%%%%%%%%%%%%%%%%%%%%%%%%%%%%%%%%%%%%%%%%%%%%%%%%%%%%%%%%%%%%%%%%%%%%%%%%%%%%%%%%%%%%%
\chapter{variable}

Python comme dit precedemment n'est pas un language typé c-a-d qu'il suffit de declarer une variable sans son type et de l'affecte directement a la valeur souhaité.\\
\underline{exemple}:\\
$varInt = 12$\\
$varFloat = 12.5$\
$varString = "StringQuelconque"$\
\\
une chaine peut etre entouré soit de guillemet \" ou d'apostrophe \' mais il faudra echapper les caractere guillement/apostrophe si celui ci figure dans la chaine .\\
Si on veux pas echapper les caratere on peux mettre la chaine en triple guillemet \verb+ """chaine d'un caractere""" +.\\

\section{Entree/Sortie}
$type(nom_var) => $ renvoie le type de la variable.\\
$print("a =", a, "et b =", b) => $ affiche la variable ou le texte sur la sortie standard.\\
$input("message : ") => $ saisie utilisateur .\\
\\
\underline{Rq}: si on veux stocker le resulatat , $varRes=input("entrez resulatat : ")$ .\\
\\

%%%%%%%%%%%%%%%%%%%%%%%%%%%%%%%%%%%%%%%%%%%%%%%%%%%%%%%%%%%%%%%%%%%%%%%%%%%%%%%%%%%%%%%%%%%%%%%

\chapter{Dictionnaire}
En python il existe des tableau associatif appelé dictionnaire et se declare de deux facons :\\
soit avec la class $dict() : mon_dic=dict()$\\
ou alors avec des parenthese : $dictionnaire = {Cles : valeur}$ \\
Les parenthèses \verb+() délimitent les tuples, les crochets [] délimitent les listes et les accolades {}+ délimitent les dictionnaires.
\\
\underline{Exemple} : \\
\begin{verbatim}
#Declaration du dictionnaire
d={"salut" : "hi" ,"Montagne":"Mountain" ,2:"Wave" ,"Manger":"eat", "Boire":"Drink"}

#Afficher tout le dictionnaire
print("\n Valeur du dictionnaire =",d,"\n")

#Affiche 
print("Salut --->",d["salut"],"\n")
\end{verbatim}
\\
\textbf{Ajout}\\
Pour ajouter une valeur dans un dictionnaire on place directement la cles et ca valeur dans le dictionnaire : $d["ajouter"]= "add" $ \\
$d={"salut" : "hi" ,"Montagne":"Mountain" ,2:"Wave" ,"Manger":"eat", "Boire":"Drink","ajouter":"add"}$ \\
\\
\\
\textbf{Supression}\\
On peux utiliser la methode $del() : del nomDict[cles] --> del d["salut"] $ \\
$ d={"Montagne":"Mountain" ,2:"Wave" ,"Manger":"eat", "Boire":"Drink","ajouter":"add"} $ \\


%%%%%%%%%%%%%%%%%%%%%%%%%%%%%%%%%%%%%%%%%%%%%%%%%%%%%%%%%%%%%%%%%%%%%%%%%%%%%%%%%%%%%%%%%%%%%%%


\chapter{Structure de controle}
\section{Operateur de controle}
\underline{Synthaxe}:\\
\\
\begin{DDbox}{\linewidth}
\begin{lstlisting}
if condition :
  print("Text")
elif condition :
  print("Text")
else :
  printf("Text")
\end{lstlisting}
\end{DDbox}{\linewidth}
\\
On peux utiliser les mot clé \textbf{OR,AND,NOT} dans l'expression de la condition.On peux aussi utiliser \textbf{in} pour savoir si la variable apartient a cette sequence ou non.\\
\underline{Exemple} :  if varString in $"abcdef" => (si varstring = a|b|c|d|e|f)$ \\


\section{Operateur de boucle}
\begin{DDbox}{\linewidth}
\begin{lstlisting}
while condition :
  #code
\end{lstlisting}
\end{DDbox}{\linewidth}
\\
\begin{DDbox}{\linewidth}
\begin{lstlisting}
for element in sequence :
  #code
\end{lstlisting}
\end{DDbox}{\linewidth}
\\
La variable "element" est créée par le \"for\", ce n\'est pas à nous de l\'instancier. Elle prend successivement chacune des valeurs figurant dans la séquence parcourue \"sequence\".

%%%%%%%%%%%%%%%%%%%%%%%%%%%%%%%%%%%%%%%%%%%%%%%%%%%%%%%%%%%%%%%%%%%%%%%%%%%%%%%%%%%%%%%%%%%%%%%

\chapter{Les fonctions}

\section{Definir une fonction}
Création d'une fonction : \\
\begin{DDbox}{\linewidth}
\begin{lstlisting}
def nom_de_la_fonction(parametre1, parametre2, parametre3, parametreN):
    # Bloc d'instructions
\end{lstlisting}
\end{DDbox}{\linewidth}

Si un parametre n'est pas saisie a l'appel d'une fonction , on peux donner une valeur par defaut a ce parametre par exemple :\\

\begin{DDbox}{\linewidth}
\begin{lstlisting}
def table(nb, max=10):
    """Table de nb jusqu a nb*max """
    i = 0
    while i < max:
        print(i + 1, "*", nb, "=", (i + 1) * nb)
        i += 1
\end{lstlisting}
\end{DDbox}{\linewidth}
\\
On peut appeler la fonction de deux façons : soit en précisant le num\'ero de la table et le nombre maximum d'affichages, soit en ne précisant que le numéro de la table (table(7)).\\
Dans ce dernier cas,max vaudra 10 par défaut.\\
\\

\chapter{Biblotheque Numpy/Scipy/Matplotlib}

Ces 3 bibliotheque permette d'utiliser des fonction mathematique , traces des graphes , faire des probabilite etc $...$ \\

\end{document}
