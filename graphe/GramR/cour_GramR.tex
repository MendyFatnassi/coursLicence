\documentclass[a4paper,12pt,openany]{book}
\usepackage [utf8]{inputenc}
\usepackage [french]{babel}
\usepackage [T1]{fontenc}
\usepackage{listings}
\usepackage{graphicx}
\usepackage{verbatim}

%coloration syntaxique
\usepackage{xcolor}
\definecolor{Zgris}{rgb}{0.87,0.85,0.85}

\newsavebox{\BBbox}
\newenvironment{DDbox}[1]{
\begin{lrbox}{\BBbox}\begin{minipage}{\linewidth}}
{\end{minipage}\end{lrbox}\noindent\colorbox{Zgris}{\\usebox{\BBbox}}
[.5cm]}

%Pour l espace entre la section et la chapitre (qui est trop grand).
\usepackage{titlesec}

\titleformat{\chapter}[block]
  {\normalfont\Huge\bfseries}% font of number
  {\chaptertitlename\ \thechapter~:}% format of number
  {20pt}% space between number and title
  {\Huge}% font of title

\titlespacing*{\chapter}
  {0pt}%  indent
  {0pt}% space before
  {20pt}% space after
\titlespacing*{\section}
  {0pt}%  indent
  {3.5ex plus 1ex minus .2ex}% space before
  {2.3ex plus .2ex}% space after

\author{Mendy Fatnassi}
\title{Théorie Des Langages}




%%%%%%%%%%%%%%%%%%%%%%%%%%%%%%%%%%%%%%	Page	%%%%%%%%%%%%%%%%%%%%%%%%%%%%%%%%%%%%%%%%

\begin{document}
\maketitle
\tableofcontents

\chapter{La Grammaire}

\section{Definition}

-\underline{Alphabet}: noté A , représente un ensemble de lettre. A={a,b,...,z} ou A2={if,then,else,id,nb,=,+}.\\
\\
-\underline{Mot}: Un mot definie une suite fini d'elment de A. exemple de mot : sur A mot 'b' , sur A2 mot "if id = nb".\\
Le mot vide se note $\epsilon$ .\\
\\
\textbf{$A^+$}: On note $A^+$ l'ensemble des mots de longueur superieur ou egale a 1.\\
\textbf{$A^*$}: On note $A^*$ l'ensemble des mots que l'on peux construire a partir de A, y compris le mot vide : $A^* = {\epsilon} \cup A^+$ \\
\\
-\underline{Langage}: Un langage, défini sur un alphabet A, est un ensemble  de mots définis sur A. Autrement dit, un langage est un sous-ensemble de $A^*$.\\


\end{document}